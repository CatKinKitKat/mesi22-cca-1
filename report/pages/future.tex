\chapter{Pesquisa futura}

A criptografia quântica é um campo em rápida evolução e há muitas áreas interessantes de pesquisa que estão sendo exploradas atualmente. Algumas das principais áreas de pesquisa futura em criptografia quântica incluem:

\begin{itemize}
  \item Desenvolver protocolos QKD mais avançados e seguros, como protocolos que usam estados quânticos de dimensão superior ou esquemas de correção de erros mais sofisticados.
  \item Melhorar o desempenho e a eficiência dos sistemas QKD, como o uso de novas tecnologias, como fotónica integrada ou fontes de fotão único, para melhorar a distância de transmissão ou a taxa de chave secreta.
  \item Investigar novas aplicações de criptografia quântica, como em redes distribuídas, computação em nuvem ou comunicação segura com satélites.
  \item Desenvolver novos métodos de defesa e mitigação contra ataques em criptografia quântica, como o uso de novas primitivas criptográficas ou técnicas de aprendizado de máquina.
  \item Explorar os limites fundamentais da criptografia quântica, como os limites da segurança ou do desempenho dos sistemas QKD, e os limites últimos da privacidade e sigilo das informações transmitidas.
\end{itemize}

\section{Perspectivas futuras}

Algumas das principais perspectivas futuras para a criptografia quântica incluem:

\begin{itemize}
  \item A implantação de redes QKD em larga escala, que podem fornecer comunicação segura para uma ampla gama de aplicações, como bancos, saúde ou governo.
  \item A integração de sistemas QKD com protocolos criptográficos clássicos, como autenticação ou protocolos de troca de chaves, para fornecer maior segurança e proteção contra ataques.
  \item O desenvolvimento de novas aplicações de criptografia quântica, como em redes distribuídas, computação em nuvem ou comunicação segura com satélites.
  \item A exploração dos limites fundamentais da criptografia quântica, tais como os limites últimos da segurança e desempenho dos sistemas QKD, e os limites últimos da privacidade e sigilo da informação transmitida.
\end{itemize}

\newpage
