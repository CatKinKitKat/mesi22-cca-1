\chapter{Princípios da Criptografia Quântica}

Os princípios da criptografia quântica são baseados nos princípios fundamentais da mecânica quântica, incluindo o princípio da incerteza, entrelaçamento e superposição. Esses princípios permitem a criação de chaves criptográficas seguras que podem ser usadas para codificar e descodificar informações transmitidas por um canal quântico.

O princípio da incerteza, formulado pela primeira vez por Heisenberg em 1927, afirma que é impossível determinar o estado exato de um sistema quântico com precisão total. Este princípio está no cerne da criptografia quântica, pois permite a criação de chaves criptográficas que são fundamentalmente seguras contra espionagem e outros ataques.

O entrelaçamento é outro princípio fundamental na criptografia quântica. Este fenómeno, descrito pela primeira vez por Einstein em 1935, refere-se ao fenómeno de dois ou mais sistemas quânticos se correlacionando de tal forma que o estado de um sistema pode ser determinado a partir do estado do outro. Na criptografia quântica, o entrelaçamento é usado para criar chaves criptográficas que são compartilhadas entre duas ou mais partes, permitindo uma comunicação segura.

A superposição é um terceiro princípio da mecânica quântica que é relevante para a criptografia quântica. Esse fenómeno, descrito pela primeira vez por Schrödinger em 1935, refere-se à capacidade de um sistema quântico existir em vários estados simultaneamente. Na criptografia quântica, a superposição é usada para codificar informações de maneira segura contra espionagem e outros ataques.

Juntos, esses princípios da mecânica quântica formam a base da criptografia quântica e permitem a criação de chaves criptográficas seguras que são fundamentalmente diferentes daquelas usadas na criptografia clássica.
