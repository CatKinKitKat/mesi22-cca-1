\chapter{Limitações}

Apesar das muitas vantagens da criptografia quântica, também existem limitações que devem ser consideradas. Essas limitações surgem das propriedades fundamentais dos sistemas quânticos e podem limitar a eficácia dos protocolos criptográficos quânticos.

Uma das principais limitações da criptografia quântica é a perda e o ruído que podem ocorrer durante a transmissão de informações por um canal quântico. Essa perda e ruído podem surgir de várias fontes, incluindo as imperfeições dos sistemas quânticos usados para transmitir as informações e os efeitos do ambiente nos estados quânticos.

A perda e o ruído que podem ocorrer em um canal quântico podem limitar a distância na qual os protocolos criptográficos quânticos podem ser usados efetivamente. Além disso, também pode reduzir a velocidade e eficiência dos protocolos, tornando-os menos práticos para muitas aplicações.

Outra grande limitação da criptografia quântica é o potencial para vulnerabilidades de segurança. Por exemplo, alguns ataques a sistemas criptográficos quânticos exploram canais laterais ou outras fraquezas no sistema quântico, permitindo que um invasor obtenha informações sem ser detectado.

Para mitigar essas limitações, os pesquisadores propuseram uma série de soluções potenciais. Essas soluções incluem o uso de técnicas avançadas de correção de erros, novos protocolos que vão além do QKD tradicional e o desenvolvimento de novas tecnologias que podem melhorar o desempenho dos sistemas criptográficos quânticos. Nas secções seguintes, exploraremos essas soluções com mais detalhes.

\section{Propostas para Mitigar as Limitações da Criptografia Quântica}

Conforme discutido na secção anterior, existem várias limitações à criptografia quântica que devem ser consideradas. Nesta secção, vamos explorar algumas das propostas que foram apresentadas para mitigar essas limitações.

Uma proposta para superar as limitações da criptografia quântica é o uso de técnicas avançadas de correção de erros. Essas técnicas podem ser utilizadas para detectar e corrigir erros que surgem durante a transmissão de informações por um canal quântico, permitindo uma comunicação mais confiável e segura.

Outra proposta é o desenvolvimento de novos protocolos que vão além do tradicional QKD. Esses protocolos, como o protocolo one-time pad (OTP), podem oferecer segurança e desempenho aprimorados em comparação com os protocolos QKD tradicionais.

Além disso, os pesquisadores também estão trabalhando em novas tecnologias que podem melhorar o desempenho dos sistemas criptográficos quânticos. Por exemplo, alguns pesquisadores estão explorando o uso de repetidores quânticos, que podem estender o alcance dos protocolos criptográficos quânticos e memórias quânticas, que podem armazenar e recuperar informações quânticas com alta fidelidade.

\newpage
